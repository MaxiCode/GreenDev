% ---------------------------------------------------------
\chapter{Background and Related Work}
\label{b_a_rel_work}
% ---------------------------------------------------------

This chapter provides an overview of the general topics of this thesis. In section \ref{background} an introduction about performance evaluation and energy consumption of configurable software is given and related work is summarized in section \ref{rel_work}.

\section{Background}
\label{background}

This section provides an introduction about necessary fundamentals. We give an overview about configurable systems in subsection \ref{background_conf_sys}. We define the terms performance \ref{background_perf} and energy consumption \ref{background_en_cons} for this thesis and describe correlation \ref{background_corr}, \todo{other definitions here}.

\subsection{Configurable Systems}
\label{background_conf_sys}

Nowadays, almost every software system can be configured by numerous options. These configurability provides a lot of flexibility. Databases for example, typically let users configure options like encryption, compression, cash size and many others. While this flexibility helps to make software customizable to the user, it also causes problems. Not only the size of the configuration space which results from the combination of configuration options is hard to grasp by users, but also possible constraints among options make configuring problematic. This may lead users to pick default configurations instead of choose a configuration that let the system perform as desired.

To describe configurable systems \fmsit\ were developed. These models describe each configuration option as a \featureit. \textit{Features} are distinct pieces of functionality which can be mandatory or optional. To group \featuresit\ that belong together \textit{or-} and \textit{alternative-groups} can be used. Usually not all combinations of \featuresit\ are valid. For example some might be required whereas others are mutually exclusive. To cope this \fmsit\ provide constraints.


\subsection{Performance}
\label{background_perf}

General, performance is the amount of work done per time\cite{tsirogiannis2010analyzing}. In software engineering, there are different performance characteristics (e.g. throughput, response time, scalability). So performance can be defines as:
\begin{equation}
\label{def:perf1}
	Performance=\frac{Completed\;Tasks}{Unit\;of\;Time}
\end{equation}
Because we want to analyze performance of software by executing benchmarks, the work done is always the same. So for this thesis we can define performance like \cite{siegmundperformance}:
\begin{equation}
\label{def:perf2}
	Performance=Execution\;Time\;of\;a\;Benchmark
\end{equation}

\subsection{Correlation}
\label{background_corr}


\section{Related Work}
\label{rel_work}

Since performance analysis of software was always an important aspect of software engineering, analyzing energy consumption of software approaches few years ago. This section summarizes recent publications in both ares. 

% Performance in software systems
\subsection{Performance Evaluation}
\label{rel_perf}

A Survey about Performance Evaluation of Component-based Software Systems \cite{koziolek2010performance}.
%A Survey about Model-based performance prediction in software development \cite{balsamo2004model}.
Early Performance Testing of Distributed Software Applications \cite{denaro2004early}.
% Software performance engineering \cite{smith1993software}.
The Future of Software Performance Engineering \cite{woodside2007future}.

