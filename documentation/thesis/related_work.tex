% ---------------------------------------------------------
\chapter{Background and Related Work}
\label{b_a_rel_work}
% ---------------------------------------------------------

This chapter provides an overview of the general topics of this thesis. In section \ref{background} an introduction about variability models, performance assessment and performance prediction is given. Related work and approaches that analyze performance simmilar to our approach is summarized in section \ref{rel_work}.

\section{Background}
\label{background}

This section provides an introduction about necessary fundamentals. We give an overview about variability models of configurable software systems in subsection \ref{background_conf_sys}. We define the term performance \ref{background_perf} for this thesis and describe how performance of software can be measured. Section \ref{background_perf_pred} describes how to learn models to be able to predict performance.


\subsection{Configurable Software Systems}
\label{background_conf_sys}

Configurations to Features - 
\cite{loughran2004framed}
However, there are no mechanisms available in current AOP languages to support and realise fine-grained configurability and variability, thus, the potential for aspects to be reused in different contexts is limited.This paper demonstrates framed aspects, an AOP independent meta language which adds the power of parameterisation, construction time constraint checking and conditional compilation to AOP languages.

Variability Model - Feature Model
model all valid configurations (in graph structure with constraints)





To describe configurable systems \fmsit\ were developed. These models describe each configuration option as a \featureit. \textit{Features} are distinct pieces of functionality which can be mandatory or optional. To group \featuresit\ that belong together \textit{or-} and \textit{alternative-groups} can be used. Usually not all combinations of \featuresit\ are valid. For example some might be required whereas others are mutually exclusive. To cope this \fmsit\ provide constraints.


\subsection{Performance}
\label{background_perf}

General, performance is the amount of work done per time\cite{tsirogiannis2010analyzing}. In software engineering, there are different performance characteristics (e.g. throughput, response time, scalability). So performance can be defines as:
\begin{equation}
\label{def:perf1}
	Performance=\frac{Completed\;Tasks}{Unit\;of\;Time}
\end{equation}
Because we want to analyze performance of software by executing benchmarks, the work done is always the same. So for this thesis we can define performance like \cite{siegmund2015performance}:
\begin{equation}
\label{def:perf2}
	Performance=Execution\;Time\;of\;a\;Benchmark
\end{equation}


\subsection{Lernability of Performance}
\label{background_perf_pred}




\section{Related Work}
\label{rel_work}

Since performance analysis of software was always an important aspect of software engineering, analyzing energy consumption of software approaches few years ago. This section summarizes recent publications in both ares. 

% Performance in software systems
\subsection{Performance Evaluation}
\label{rel_perf}

A Survey about Performance Evaluation of Component-based Software Systems \cite{koziolek2010performance}.
%A Survey about Model-based performance prediction in software development \cite{balsamo2004model}.
Early Performance Testing of Distributed Software Applications \cite{denaro2004early}.
% Software performance engineering \cite{smith1993software}.
The Future of Software Performance Engineering \cite{woodside2007future}.

