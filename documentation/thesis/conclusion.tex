\chapter{Conclusion}
\label{chap:conlusion}

\section{Concluding Remarks}

With this thesis, we present an approach for automating the currently static performance analysis process of configurable software systems. This work provides reasonable guidelines and issues to each stage of the performance analysis process. We surveyed different monitoring tools and sampled the configuration space of our subject systems. We utilized existing techniques to build performance influence models for methods and reason about the used regression techniques and analyzed sources of bias in of our approach. 

We found out that we can reduce performance monitoring time by excluding at least 30\% of the methods. If we concentrate only on those methods with the highest configuration sensitivity (that we could not learn with our models), we can even reduce the time to monitor performance by 94\%. 
Profiling needs at least triples the time so we can achieve at least triple measuring speed, if we do the data flow analysis apart from the profiling process. Therefore, we could use Kieker or SPASS--meter to speed up.
To integrate the results of our approach to the \ac{SPE} process we make use of novel visualisations that present detected hot-spots integrated into existing development tooling. Finally, we can say that we understand the behaviour of performance hot-spots of configurable software systems better than before.


\section{Outlook and Future Work}

%Was könnte passieren damit?
%Wa muss man dafür tun?

Most effort should be spent in the analysis of methods that are strongly sensitive to configurations, because these methods characterize a huge fraction of the overall performance of software systems. 
Therefore, researchers could adapt the sampling strategy. To determine which features influence which method, we could use feature wise sampling of configuration options. 
Another issue of in our approach might be that we only analyzed individual features and pair-wise interactions. Through the discussion with the Catena developer, we know that Garlic, Lambda, and the chosen Hash function are the main reason for the performance of the graphs. So, in an reproduction of this study, at least interactions among three features should be considered.
We analyzed the overall measurement bias to the performance measurements. But until now, it remain unclear which proportion the individual sources of non-determinism have. We suppose the \ac{GC} to influence the performance substantial. For clarification, the activity of the \ac{GC} could be logged during the measurements.
We choose to measure the performance of 1000 configurations for each of the three subject systems. We do not differentiate between the size of the projects and the number of methods to be analyzed. With Catena, we decide between four different graphs, which results in 250 measurements per graph. In further experiments, the focus should be on a proper sample size, which provides enough data to be able to learn the influence models.
Finally, to be able to generalize the results, other configurable software systems, developed in Java or other languages, should be analyzed as well. 