%\documentclass[german,bachelor]{webisthesis}
\documentclass[english,master]{webisthesis}
% When you change the language, pdflatex may halt on recompilation.
% Just hit enter to continue and recompile again. This should fix it.

%
% Values
% ------
\ThesisSetTitle{Supporting the Discovery of Performance Hot-Spots}
\ThesisSetKeywords{Performance, Configurable Software Systems} % only for PDF meta attributes

\ThesisSetAuthor{Max Weber}
\ThesisSetStudentNumber{110068}
\ThesisSetDateOfBirth{04}{05}{1991}
\ThesisSetPlaceOfBirth{Jena}

\ThesisSetSupervisors{Prof.\ Dr.\ Ing.\ Norbert Siegmund, Junior\ Professor\ Dr.\ Florian\ Echtler}

\ThesisSetSubmissionDate{22}{5}{2018}

%
% Suggested Packages
% ------------------
%\usepackage{harvard}
\usepackage{natbib}
%\usepackage[round]{natbib}
\bibliographystyle{plainnat}
%S\setcitestyle{authoryear,open={[},close={]}}
%\bibliographystyle{plain}

%   Allows citing in different ways (e.g., only the authors if you use the
%   citation again within a short time).
%
\usepackage{graphicx}
\usepackage{booktabs}
\usepackage{listings}
\usepackage{pdflscape}
\usepackage{makecell}
%\usepackage{tabu}
\usepackage{rotating}
\usepackage{enumitem}
%    For tables ``looking the right way''.
%
%\usepackage{tabularx}
%    Enables tables with columns that automatically fill the page width.
%
%\usepackage[ruled,algochapter]{algorithm2e}
%    A package for pseudo code algorithms.
%
%\usepackage{amsmath}
%    For tabular-style formatting of mathematical environments.
%

% Create links in the pdf document
% Hyperref has some incompatibilities with other packages
% Some other packages must be loaded before, some after hyperref
% Additional options to the hyperref package can be provided in the braces []
\usehyperref[backref] % This will add back references in the bibliography

\RequirePackage{xcolor}
\newcommand{\todo}[1]{\colorbox{red}{\color{white}\sffamily TODO: #1}}

\begin{document}
\begin{frontmatter}
  \begin{abstract}
%During the past years the information and communication technologies (ICT) sector grew exponentially. It is to be expected that this trend continues in the next years. Currently, the energy needed to supply data centers, servers and every kind of computational device increases with the progress of the ICT sector. This poses economic and environmental complications.

%To have energy available it has to be generated, stored and delivered. Storage requires materials for batteries, delivery requires infrastructure and also a big issue in this context is cooling down the hardware. For years, economy tries to save energy consumption of hardware (e.g. European Union energy label). But while hardware primarily consumes energy, it can only be as efficient as the software that operates that hardware.

%This master thesis proposes and implements a tool to analyze software regarding their energy consumption. In relation to other research in this area, this thesis tries to answer the following questions. Is it possible to identify energy leaks in software? Does configuration of software affect energy consumption? Is it possible to predict energy consumption of software with unknown configurations? And finally, does visualization of energy hot spots in source code help programmers to rectify them?

%To answer this questions we want to analyze performance and energy consumption of software projects written in Java. We want to profile the project on the one hand with different configurations to identify configuration sensitivity and on the other hand with different inputs to identify input sensitive parts. Next, we want to relate performance to the energy consumption. After that we want to investigate up to which level it is possible to predict energy consumption and performance of untested configurations.

%Finally, we want to validate our work by showing that our approach can automatically detect and rank energy leaks in software projects for which such leaks are known. To verify that out tool helps software developer to rectify energy hot spots in source code we will conduct a user study with and without help of the tool.

%If our results are satisfying we have created a tool which helps software developer to be aware of their of their energy footprint. So there will be more energy efficient programs that will have less energy costs.
  \end{abstract}

  \tableofcontents

  %\chapter*{Acknowledgements} % optional
  %I thank the authors of the webisthesis template for their excellent work!

  %\listoffigures % optional

  %\listoftables % optional

  %\listofalgorithms % optional
  %    requires package algorithm2e

  % optional: list of symbols/notation (e.g., using the nomencl package)
\end{frontmatter}

% ---------------------------------------------------------
% Main content
% ---------------------------------------------------------

\pagenumbering{arabic}
%\onehalfspacing
% ---------------------------------------------------------
\chapter{Introduction}
\label{chap:introduction}
% ---------------------------------------------------------

Since the beginning of software engineering developers try to produce software which performs well. Over the years software has to run more difficult, complex and computing-intensive tasks. Hardware, software and the technology to work with grows and becomes more and more complex. Anytime software has to fulfill different quality properties. This includes aspects like reliability, security, maintainability and \textit{performance}. Each aspect is analyzed and described in many publications. With the analysis of reliability one tries to reduce and prevent application problems that affect users. Also security and maintainability are subjects of research for decades (\cite{liskov1972design,johnstone1978computer,yau1980some}). The aspect efficiency of software systems describes how source code and software architecture perform in run-time scenarios. Since the beginning of software engineering \textit{performance} is one of the most important aspects.\\
This thesis will analyze \textit{performance} of software. With respect to software and software systems there are different definitions of \textit{performance} in literature. While \textit{performance} from an end users perspective describes the ability that users can perform a certain task without perceivable delay or without irritation \cite{Molyneaux:2009:AAP:1550832}, the developers perspective is explained by so called non-functional properties. This properties provide a way to judge about the quality of software systems. This was already done by other researchers \cite{Molyneaux:2009:AAP:1550832,weber2005key,siegmund2015performance,smith1993software}. The main difference from this work to others is that we analyzed the \textit{performance} of software projects on method level, which cause different challenges (described in chapter \ref{perf_measure}).\\
Nowadays, almost every software system can be configured by numerous options. These configurability provide the possibility to customize the programs behavior to the users needs. Databases for example, typically let users configure options like encryption, compression, cash size and many others. While this flexibility helps to make software customizable to the user, it also causes an increased effort during testing, maintaining and analyze of \textit{performance}. Not only the size of the configuration space which results from the combination of configuration options is hard to grasp by users. Because programs with more than 265 configuration options provide more different configurations than atoms in the universe, which make testing and analyzing them all unpracticable. But also possible constraints among options make configuring problematic. This may lead users to pick default configurations instead of choosing a configuration that let the system perform as desired. Therefore we will sample configurations to get an subset which describes the whole configuration space for our analysis, so we do not have to analyze them all.


% roughly describe way to detect performance hot spots on function level
% specialization of this work
% ------------GOAL---------------- 
To achieve our goal, we want to analyzes \textit{performance} of software systems on method level. To understand the influence of configuration options on the \textit{performance} of methods we profile each subject system with an appropriate profiler to extract the runtime of single functions of the running program. As we decided to take JAVA projects fro our subject systems, we also want to analyze the influence of the environment in which the test are running. An visualization of our findings direct in the source code will help future developers identify possible \textit{performance} hot-spots. An overview of the \textit{performance} distribution over all functions in an project gives new perspectives to support \textit{performance} analysis.\\

% ------------Approach----------------

To analyze \textit{performance} we follow a five step approach:

\begin{enumerate}
	\item Identification of whole configuration space and determining different workload levels.
	\item Profiling the execution of software projects with sampled configurations and different workloads.
	\item Processing extracted data to assess \textit{performance} on function level and to determine significance of the data.
	\item Creating \textit{performance} influence models.
	\item Learning influence of configurations on functions of the software.
	\item Visualize \textit{performance hot-spots} direct in the source code and create suitable visualizations of performance dependencies.
\end{enumerate}


% ------------Contribution------------

This thesis provides different contributions. First, we provide an detailed analysis of the influence on performance of the environment on which the software runs. We also compare different sampling strategies to evaluate coverage of the configuration space. Then we analyze the learnability of performance on method level influenced by configurations. We also provide a overview-visualization of the influence of configurations on performance. Finally, we developed an Eclipse-Plug-in to visualize performance hot-spots direct in the source code.
% direct in source code 
% provide an overview of performance of configurable software systems

% Analyze performance and energy consumption of Software Projects written is JAVA.
% Analyze performance on function level
% Vary projects configuration to identify configuration sensitivity
% Vary projects input to identify input sensitive parts
% Relate energy consumption of the system to parts of the software
% Predict energy consumption and performance of untested configurations
% ---------------------------------------------------------
\chapter{Background and Related Work}
\label{b_a_rel_work}
% ---------------------------------------------------------

This chapter provides an overview of the general topics of this thesis. In section \ref{background} an introduction about variability models, performance assessment and performance prediction is given. Related work and approaches that analyze performance simmilar to our approach is summarized in section \ref{rel_work}.

\section{Background}
\label{background}

This section provides an introduction about necessary fundamentals. We give an overview about variability models of configurable software systems in subsection \ref{background_conf_sys}. We define the term performance \ref{background_perf} for this thesis and describe how performance of software can be measured. Section \ref{background_perf_pred} describes how to learn models to be able to predict performance.


\subsection{Configurable Software Systems}
\label{background_conf_sys}

Configurations to Features - 
\cite{loughran2004framed}
However, there are no mechanisms available in current AOP languages to support and realise fine-grained configurability and variability, thus, the potential for aspects to be reused in different contexts is limited.This paper demonstrates framed aspects, an AOP independent meta language which adds the power of parameterisation, construction time constraint checking and conditional compilation to AOP languages.

Variability Model - Feature Model
model all valid configurations (in graph structure with constraints)





To describe configurable systems \fmsit\ were developed. These models describe each configuration option as a \featureit. \textit{Features} are distinct pieces of functionality which can be mandatory or optional. To group \featuresit\ that belong together \textit{or-} and \textit{alternative-groups} can be used. Usually not all combinations of \featuresit\ are valid. For example some might be required whereas others are mutually exclusive. To cope this \fmsit\ provide constraints.


\subsection{Performance}
\label{background_perf}

General, performance is the amount of work done per time\cite{tsirogiannis2010analyzing}. In software engineering, there are different performance characteristics (e.g. throughput, response time, scalability). So performance can be defines as:
\begin{equation}
\label{def:perf1}
	Performance=\frac{Completed\;Tasks}{Unit\;of\;Time}
\end{equation}
Because we want to analyze performance of software by executing benchmarks, the work done is always the same. So for this thesis we can define performance like \cite{siegmund2015performance}:
\begin{equation}
\label{def:perf2}
	Performance=Execution\;Time\;of\;a\;Benchmark
\end{equation}


\subsection{Lernability of Performance}
\label{background_perf_pred}




\section{Related Work}
\label{rel_work}

Since performance analysis of software was always an important aspect of software engineering, analyzing energy consumption of software approaches few years ago. This section summarizes recent publications in both ares. 

% Performance in software systems
\subsection{Performance Evaluation}
\label{rel_perf}

A Survey about Performance Evaluation of Component-based Software Systems \cite{koziolek2010performance}.
%A Survey about Model-based performance prediction in software development \cite{balsamo2004model}.
Early Performance Testing of Distributed Software Applications \cite{denaro2004early}.
% Software performance engineering \cite{smith1993software}.
The Future of Software Performance Engineering \cite{woodside2007future}.


% ---------------------------------------------------------
\chapter{Big Picture}
\label{chap:method}
% ---------------------------------------------------------

\section{Monitoring of Conf Software}


\subsection{On Function Level}


\section{Learnability}
% ---------------------------------------------------------
\chapter{Methodology}
\label{chap:p_measurement}
% ---------------------------------------------------------

 Measurement of performance is an important aspect of all kinds of science. 
% performance measurement of java 
% performance term is explained in chap Background
% how does mesurement of performance work in general?
% Impact from OS and Hardware (CPU overclock, memory access, memory level)
% Sampling Strategies


\section{Test Environment}
\label{test_env}


\section{Case Studies}
\label{case_studies}

Eingrenzung auf Java software because, widespread, popular, often used, runs on billions of devices

% 3 Java projects 
% 3 different kinds of projects to depict variety
% Configurable, Workload of projects to manage execution time and to validate results of projects with different workloads
% different configuration space 
% different ammount of configuration options
% 
% Definition of configuration space for subject systems\cite{Han:2016:ESP:2961111.2962602}:
% collection of information with studying all artifacts that are publicly available to users including documents (e.g. user manuals and online help pages), configuration files, and source code. -> anoter reason for open sourece projects because else there were only provided configurations available 

% proveded performance tests for the case studies
% Configuration space definition(\cite{Han:2016:ESP:2961111.2962602}): they also described the configuration space not only by parameter, but also with hidden configuration options and system-specific configurations
% they also described the configuration space not only by parameter, but also with hidden configuration options and system-specific configurations
% -> This thesis: system configuration options are restricted to one system to be able to limit the overall configuration options
% configuration parameter cause 78%-92% of configuration-related performance bugs
% for this thesis we do not investigate 8%-17% of system-level configuration bugs

\subsection{Catena}

 Project description
% Project size in Classes and Functions
% Project properties (SPL Model)
% Features explained
% Workload

\subsection{H2}

Project description
% Project size in Classes and Functions
% Project properties (SPL Model)
% Features explained
% Workload

\subsection{Sunflow}

Project description
% Project size in Classes and Functions
% Project properties (SPL Model)
% Features explained
% Workload

\section{Overall Process}
\label{overall_process}

\begin{figure}
  \centering
  \includegraphics[width=0.7\textwidth]{images/workflow_overview_0}
  \caption{Performance Analysis Workflow.}
  \label{big_picture_perf_anal_workflow}
\end{figure}

% Overview of the process
\todo{Some information from the following paragraph might be shifted to chap. \ref{chap:p_measurement}}
We divided our approach into three parts as seen in Figure~\ref{big_picture_perf_anal_workflow}. First, we extract the performance data from the subject systems we used (see section~\ref{perf_extr}). Therefore, we run them in our test environment, which we describe in section~\ref{test_env}. We execute each of them with different configurations and workloads. We sampled the space of all possible configuration as described in the next section (section~\ref{perf_measure_sampling}) During execution, they were profiled to extract the performance of each method of the projects which was executed (we surveyed available performance monitoring tools in section~\ref{monitoring_tools}). In the second step, which is described more precisely in section~\ref{data_prozessing}, we process the collected data. We also trained a regression model that aims on predicting the performance of each method depending on given configurations. The last step of our approach includes visualizations of performance data (described in section~\ref{visualisation}). We designed an Eclipse plugin to visualize performance hot spots through the \ac{IDE} directly in the source code. Additionally, we used a graph-based visualisation to given an overview of the influence if configuration options on the performance of software systems.


\subsection{Performance extraction}
\label{perf_extr}

\begin{figure}
  \centering
  \includegraphics[width=0.7\textwidth]{images/workflow_perf_expanded}
  \caption{Performance Extraction Worklow.}
  \label{perf_extr_workflow}
\end{figure}


This section deals with performance extraction from configurable software systems. Figure \ref{perf_extr_workflow} shows our procedure. The orange parts of this diagram represent existing software and tools we used for this thesis, the green parts are results of our work. We choose three Java projects from GitHub as case studies. To be able to run software, it has to be initialize with parameters (configuration) and executed with a task (workload). Depending on the software, there are many different configurations to choose from. Also the number of different workloads depends on the software and the expected output. A general description of the projects, the space of possible configurations, and workloads was explained in section~\ref{case_studies}. Besides a configuration and workload, software needs an environment in which it can be executed. The environment can be defined by hardware, \ac{OS}, software and system settings. In order to measure performance during execution, we used the \ac{JIP} (section~\ref{perf_measure} compares different Java profiler). This profiler is able to measure the run time of each method during execution of the software. However, we have to deal with the influence of non-determinism that affect overall performance. Our measurements run on real hardware (test environment is described in section~\ref{test_env}) in a \ac{JRE} with other processes being executed in parallel. 

%Because software runs on real hardware, the measurements might be influenced by a measurement bias. Hence, we want to find out if and to which degree the environment has an influence on performance. We also trained a regression model that aims on predicting the performance of each method depending on given configurations. An introduction to linear regression and an overview of the tools we used for modelling can be found in section~\ref{lin_reg}. 

\subsubsection{System Influences}

separate machine
Multithreading
Only needed software installed
5 seconds delay between individual measurements

%Java specific performance measurement concerns~\ref{perf_measure}

%\subsection{Monitoring/Profiling}
%pitfalls
%strategies 
%solutions
%expected outcome
%A detailed analysis of available profiling tools is provided in section~\ref{perf_measure}. 


%Extracting performance of software. 
%Software is configurable.
%Identification of configuration options.
%Profile them with some workload.

%Performance of software is defined as described in~\ref{def:perf2}.

%Sources of non-determinism whole measuring performance.
%Environment in which software is executed is defined by hardware, \ac{OS}, software and system settings.
%Processes and threading is also a matter if on multi core machine.
%Because we want to analyze Java projects we have to take care about \ac{JIT}, \ac{GC} and \ac{JVM}.

\subsubsection{Java Mesurement Influences}
\label{perf_measure}

Java specific impacts on program execution and how to overcome them
% Related paper 
% Garbage collector
% Java Byte code optimization


\subsection{Data Processing}
\label{data_prozessing}

\begin{figure}
  \centering
  \includegraphics[width=0.7\textwidth]{images/workflow_data_expanded}
  \caption{Data Processing Worklow.}
  \label{perf_extr_workflow}
\end{figure}

\subsubsection{Relative Performance}

\subsubsection{Measurement Influence}


\subsection{Data Visualisation}
\label{visualisation}

\begin{figure}
  \centering
  \includegraphics[width=0.7\textwidth]{images/workflow_visual_expanded}
  \caption{Visualisation Worklow.}
  \label{perf_extr_workflow}
\end{figure}

\subsubsection{Eclipse Plugin Data}

\subsubsection{Flame Graph Data}


\chapter{Evaluation}
\label{chap:analysis}

%- RQs

\section{Measurement Setup}

Reproducability

\section{Results}

Present raw results

% use quality metrics to describe
% Impact of configuration
% Impact of workload

\section{Discussion}
\label{discussion}

% - answer RQs

\section{Detailed Analysis}

% - concentrate on Interesting parts
% - with help of tools

\subsection{GC Analysis}
\label{gc_analysis}

\subsection{Eclipse Plugin}
\label{eclipse_plugin}

\subsection{FlameGraphs}
\label{flame_graph}



\section{Threads on Validity}

\subsection{Internal Validity}

\subsection{External Validity}
\chapter{Conclusion}
\label{chap:conlusion}

\section{Concluding Remarks}

\section{Outlook and Future Work}


% ---------------------------------------------------------
% References and Appendix
% ---------------------------------------------------------

\bibliography{bib}
% ---------------------------------------------------------
\chapter*{Acronyms}
% ---------------------------------------------------------

% ---------------------------------------------------------
% Put longest word in square brackets for column spacing
% ---------------------------------------------------------

\begin{acronym}[WWWW]
  \acro{JVM}{Java virtual machine}
  \acro{KPI}{Key Performance Indicators}
  \acro{IDE}{Integrated Development Environment}
  \acro{SPL}{Software Product Line}
  \acro{NFR}{non-functional requirements}
  \acro{NFP}{non-functional properties}
  \acro{SPE}{Software Performance Engineering}
  \acro{CART}{Classification-And-Regression-Tree}
  \acro{OSS}{Open-source Software}
  \acro{OS}{Operating System}
  \acro{JIT}{Just-In-Time}
  \acro{GC}{Garbage Collection}
  \acro{JIP}{Java Interactive Profiler}
  \acro{JRE}{Java Runtime Environment}
  \acro{GUI}{Graphical User Interface}
  \acro{AOP}{aspect-oriented programming}
\end{acronym}

\appendix
% ---------------------------------------------------------
\chapter{Appendix}
\label{appendix}
% ---------------------------------------------------------

\section{Feature Diagrams}
\label{feature_diagrams}

\subsection{Running Example}

\lstinputlisting[basicstyle=\scriptsize\ttfamily, captionpos=b, language=xml, caption={Textual representation of example database model.}, label={app_feature_diagrams}]{images/example_database.xml}

\subsection{Case Studies}


\section{Scripts}

\lstinputlisting[basicstyle=\scriptsize\ttfamily, captionpos=b, language=bash, caption={Script for enabling/disabling Turbo Boost in Ubuntu (\cite{turboBoost2015maythux}).}, label={script_overclocking}]{images/turbo_boost.sh}


\section{Performance Log File}
\lstinputlisting[basicstyle=\scriptsize\ttfamily, captionpos=b, caption={Example Log File of Catena Password Hashing Framework.}, label={example_log_file}]{images/example_log_file.txt}

% ---------------------------------------------------------
%%\pagestyle{scrheadings}             % Überschriften in den Kopfzeilen
%\tableofcontents
\addcontentsline{toc}{chapter}{\listfigurename}
\listoffigures                      % Abbildungsverzeichnis
\addcontentsline{toc}{chapter}{\listtablename}
\listoftables                       % Tabellenverzeichnis
% \printnomenclature
% \addcontentsline{toc}{chapter}{\nomenclaturename}


\end{document}

