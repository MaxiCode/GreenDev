% ---------------------------------------------------------
% Theorems
% ---------------------------------------------------------

%\newtheorem{theorem}{Theorem}[chapter]
%\newtheorem{lemma}[theorem]{Lemma}
%\newtheorem{proposition}[theorem]{Proposition}
%\newtheorem*{corollary}{Corollary}
%\newtheorem{definition}{Definition}[chapter]
%\newtheorem{conjecture}{Conjecture}[chapter]
%\newtheorem{example}{Example}[chapter]
%\newtheorem{assumption}{Assumption}[chapter]
%
%\theoremstyle{remark}
%\newtheorem*{remark}{Remark}
%\newtheorem*{note}{Note}
%\newtheorem{case}{Case}

% ---------------------------------------------------------
% Graphic paths
% ---------------------------------------------------------

\DeclareGraphicsExtensions{.pdf,.png,.jpg}
\newcommand{\figurepath}{images}

% ---------------------------------------------------------
% Spacing
% ---------------------------------------------------------

%\renewcommand{\arraystretch}{1.15}
%
%\setlength{\parindent}{0pt}
%\setlength{\parskip}{1.0ex plus 0.0ex minus 0.0ex}
%
%\setheadsepline[\textwidth]{1pt}
\setcounter{tocdepth}{1}

% Define LuciusWhite colorscheme
\definecolor{luciusBackground}{HTML}{FFFFFF}
\definecolor{luciusForeground}{HTML}{444444}
\definecolor{luciusBlack}{HTML}{444444}
\definecolor{luciusRed}{HTML}{AF0000}
\definecolor{luciusGreen}{HTML}{008700}
\definecolor{luciusOrange}{HTML}{AF5F00}
\definecolor{luciusBlue}{HTML}{005FAF}
\definecolor{luciusPurple}{HTML}{870087}
\definecolor{luciusCyan}{HTML}{008787}
\definecolor{luciusWhite}{HTML}{FFFFFF}

% Background color from LuciusLight
\definecolor{luciusGray}{HTML}{EEEEEE}

% Additional colors
\definecolor{lightGray}{HTML}{ECECEC}
\definecolor{mediumGray}{HTML}{D9D9D9}
\definecolor{darkGray}{HTML}{999999}

% ---------------------------------------------------------
% Listings
% ---------------------------------------------------------
 
%define Javascript language
\lstdefinelanguage{JavaScript}{
keywords={typeof, new, true, false, catch, const, function, return, null, catch, switch, let, var, if, in, while, do, else, case, break},
keywordstyle=\color{luciusBlue}\bfseries,
ndkeywords={class, export, boolean, throw, implements, import, this, Catena, UintX, StringEncode},
ndkeywordstyle=\color{darkGray}\bfseries,
identifierstyle=\color{luciusBlack},
sensitive=false,
comment=[l]{//},
morecomment=[s]{/*}{*/},
commentstyle=\color{luciusGreen}\ttfamily,
stringstyle=\color{luciusRed}\ttfamily,
morestring=[b]',
morestring=[b]"
}

\lstset{
language=JavaScript,
extendedchars=true,
basicstyle=\footnotesize\ttfamily,
showstringspaces=false,
showspaces=false,
numbers=left,
numberstyle=\footnotesize,
numbersep=9pt,
tabsize=2,
breaklines=true,
showtabs=false,
captionpos=b
}

%define C++ language
\lstdefinelanguage{Cpp}{
keywords={const, size_t, uint8_t, uint16_t, void},
keywordstyle=\color{luciusBlue}\bfseries,
ndkeywords={class, this, MALLOC, Catena},
ndkeywordstyle=\color{darkGray}\bfseries,
identifierstyle=\color{luciusBlack},
sensitive=false,
comment=[l]{//},
morecomment=[s]{/*}{*/},
commentstyle=\color{luciusGreen}\ttfamily,
stringstyle=\color{luciusRed}\ttfamily,
morestring=[b]',
morestring=[b]"
}

\lstset{
language=Cpp,
extendedchars=true,
basicstyle=\footnotesize\ttfamily,
showstringspaces=false,
showspaces=false,
numbers=left,
numberstyle=\footnotesize,
numbersep=9pt,
tabsize=2,
breaklines=true,
showtabs=false,
captionpos=b
}

%\renewcommand{\lstlistlistingname}{Verzeichnis der Listings}
%\definecolor{darkgreen}{rgb}{0,.4,0}
%\definecolor{darkviolett}{rgb}{.4,0,.4}
%\definecolor{blue}{rgb}{0,0,.9}
%\newcommand{\listingswidth}{\textwidth-1cm}
%\lstset{
%    language=Java,                  % oder  C++, Pascal, {[77]Fortran}, ...
%    numbers=left,                   % Position der Zeilennummerierung
%    firstnumber=auto,               % Erste  Zeilennummer
%    basicstyle=\ttfamily\footnotesize,     % Textgröße  des Standardtexts
%    keywordstyle=\ttfamily\bfseries\color{blue},    % Formattierung Schlüsselwörter
%    commentstyle=\ttfamily\color{gray},       % Formattierung Kommentar
%    stringstyle=\ttfamily\color{darkgreen},             % Formattierung Strings
%    numberstyle=\small,              % Textgröße der Zeilennummern
%    stepnumber=1,                   % Angezeigte Zeilennummern
%    numbersep=5pt,                  % Abstand zw. Zeilennummern und Code
%    aboveskip=15pt,                 % Abstand oberhalb des Codes
%    belowskip=11pt,                 % Abstand unterhalb des Codes
%    captionpos=b,                   % Position der Überschrift
%    linewidth=\textwidth,               % \textwidth-2em
%    xleftmargin=10pt,               % Linke Einrückung
%    frame=lines,                   % Rahmentyp
%    breaklines=true,                % Umbruch langer Zeilen
%    showstringspaces=true,          % Spezielles Zeichen für Leerzeichen
%    tabsize=2,
%    escapeinside={@}{@)}			% Escaping labels
%}
\lstdefinelanguage{Rust}
{
  alsoletter={=>!&-},
  morekeywords={
          as, break, const, continue, crate, else, enum, extern, false, fn, for,
          if, impl, in, let, loop, match, mod, move, mut, pub, ref, return,
          Self, self, static, struct, super, trait, true, type, unsafe, use,
          where, while,
          abstract, alignof, become, box, do, final, macro, offsetof, override,
          priv, proc, pure, sizeof, typeof, unsized, virtual, yield,
          bool, char, i8, i16, i32, i65, u8, u16, u32, u64, isize, usize, f32,
          f64, str, String
          =>, &self, enum
  },
  % macros
  emph={println!,vec!,macro_rules!,assert_eq!},
  sensitive=true, % keywords are case-sensitive
  morecomment=[l]{//}, % l is for line comment
  morecomment=[l]{///}, % l is for line comment
  morecomment=[l]{//!}, % l is for line comment
  morestring=[b]" % defines that strings are enclosed in double quotes
}

\lstset{
  basicstyle=\small\ttfamily, % Global Code Style
  captionpos=b, % Position of the Caption (t for top, b for bottom)
  extendedchars=true, % Allows 256 instead of 128 ASCII characters
  tabsize=2, % number of spaces indented when discovering a tab
  columns=fixed, % make all characters equal width
  keepspaces=true, % does not ignore spaces to fit width, convert tabs to spaces
  showstringspaces=false, % lets spaces in strings appear as real spaces
  breaklines=true, % wrap lines if they don't fit
  % frame=trbl, % draw a frame at the top, right, left and bottom of the listing
  % frameround=tttt, % make the frame round at all four corners
  framesep=4pt, % quarter circle size of the round corners
  numbers=left, % show line numbers at the left
  numbersep=3pt,                   % how far the line-numbers are from the code
  numberstyle=\tiny\ttfamily, % style of the line numbers
  commentstyle=\color{luciusGreen}, % style of comments
  keywordstyle=\color{luciusPurple}, % style of keywords
  stringstyle=\color{luciusBlue}, % style of strings
  emphstyle=\color{luciusRed}, % highlight macros
  % backgroundcolor=\color{luciusGray},
}

\lstdefinestyle{nonumbers}{
  numbers=none,
}
% ---------------------------------------------------------------------
%     Headings
% ---------------------------------------------------------------------

% \let\Chaptermark\chaptermark
% \def\chaptermark#1{
%  \def\Chaptername{#1}\Chaptermark{#1}
%  \markright{\chaptermarkformat\Chaptername \hfill}}
% \let\Sectionmark\sectionmark
% \def\sectionmark#1{
%  \def\Sectionname{#1}\Sectionmark{#1}
%  \markright{\chaptermarkformat\Chaptername \hfill
%    \sectionmarkformat\Sectionname}}


% ---------------------------------------------------------
% Links and Hyperlinks
% ---------------------------------------------------------

\hypersetup{
    breaklinks=true,           % Zeilenumbruch bei Links
    colorlinks=true,           % true = farbige Links, false = Rahmen
    linkcolor=blue,            % Farbe für Links auf gleicher Seite
    citecolor=blue,            % Farbe für Links auf Zitatstellen
    filecolor=blue,
    urlcolor=blue,             % Farbe für Links (nicht nur URLs)
    anchorcolor=blue,          % Farbe für Anker (?)
    menucolor=red,             % Farbe für Acrobatmenüeinträge
    plainpages=false           % Fehlermeldungen reduzieren
}

% ---------------------------------------------------------
% Hyphenation
% ---------------------------------------------------------

% \touchttfonts{} % hyphenation for texttt

% ---------------------------------------------------------
% Itemize and Enumeration Labels
% ---------------------------------------------------------

% \setlist[itemize,1]{label={\small$\bullet$}}
\setlist[itemize,2]{label={\tiny$\bullet$}}
\setlist[itemize,3]{label={\huge$\cdot$}}
\setlist[itemize,4]{label={\LARGE$\cdot$}}
