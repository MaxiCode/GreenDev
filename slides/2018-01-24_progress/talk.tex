\documentclass[11pt,aspectratio=169]{beamer}

% ---------------------------------------------------------------------
% Preamble
% ---------------------------------------------------------------------

%--------------------------------------------------------------------
% Pakete
%--------------------------------------------------------------------
\usepackage[T1]{fontenc}       % Korrektes Trennen von Wörtern mit Umlauten
\usepackage[english]{babel}    % Neue deutsche Silbentrennung
\usepackage{hyperref}
\usepackage{graphicx}          % Grafikeinbindung
\usepackage[dvipsnames]{xcolor}
%\usepackage{acro}
\usepackage[withpage]{acronym} % Abkürzungen verwalten
\usepackage{enumitem}          % Listen flexibel anpassen, z.B. keine Einrückung
\usepackage{listings}          % Quellcode einbinden und formatieren
\usepackage[numbers,sort&compress]{natbib} % Zitieren nach [Autor, Jahr] und [1]
\usepackage[zerostyle=d]{newtxtt}
\usepackage{hyphenat}
\usepackage{tikz}
\usepackage{float}
\usepackage{xspace}
\usepackage{lstautogobble}
\usepackage[toc,page]{appendix}
\usepackage[euler]{textgreek} % greek symbols without math environment;
                              % requires texlive-langgreek

\setlength\parindent{0pt}

%\usepackage{epsfig, psfrag}
%\usepackage[multiple]{footmisc}
%
%\usepackage{mdframed}
%
%\usepackage{algorithm}
%%\usepackage{algorithmicx}
%\usepackage{algpseudocode}
%\usepackage{amsmath}
%\usepackage{amssymb}                % Erweiterter Mathematiksatz
%\usepackage{amsthm}
%\usepackage{booktabs}               % Schöne Tabellen mit \toprule\midrule\bottomrule
%\usepackage{calc}                   % Rechnen in LaTeX
%\usepackage[font=small,labelfont=it,format=hang]{caption}[2008/04/01]  %
%\usepackage[right]{eurosym}         % Eurosymbol mit \euro und \EUR{Zahl}
%\usepackage{float}                  % Gleitobjekte fixieren
%\usepackage{footnote}
%\usepackage[left=3cm,right=3cm,top=3cm,bottom=3cm]{geometry}
%%\usepackage[colorlinks=false]{hyperref}             % Links
%\usepackage{hyphenat}               % Umbruch
%\usepackage{ifpdf}
%\usepackage{ifthen}                 % if-Verzeigung
%\usepackage{lmodern}
%\usepackage{multirow}               % Tabellenzelle über mehrere Zeilen
%\usepackage{nomencl}
%\usepackage{paralist}               % Kompakte Listen (Aufzählungen)
%\usepackage{scrpage2}               % KOMA: Kopf- und Fußzeilen ändern
%\usepackage{setspace}
%\usepackage{xspace}
%% Bei Verwendung von lmodern fehlen ein paar Zeichen, z.B. Punkt bei itemize
%% --> Font Warning: Font shape `OMS/lmr/m/n' undefined
%% --> Some font shapes were not available, defaults substituted.
%% Deshalb Paket 'textcomp' laden.
%\usepackage{shortvrb}
%\usepackage{tabularx}               % Tabelle mit fester Gesamtbreite und
%                                    % variabler Spaltenbreite
%\usepackage{textcomp}               % Zusätzliche Sonderzeichen
%%\usepackage[table]{xcolor}          % Erweitertes Farbpaket mit vielen Farbmodellen
%\usepackage{xtab}

%\usepackage[zerostyle=d]{newtxtt}
%\usepackage{tikz}
%\usepackage{float}

\usepackage{fancyvrb}   % BVerbatim

\graphicspath{{./images}}

% ---------------------------------------------------------------------
% Commands
% ---------------------------------------------------------------------

\newcommand{\eg}{\textit{e.g.,}\xspace}
\newcommand{\shortsep}{||}
\newcommand{\sep}{\ ||\ }

\newcommand{\hf}{\ensuremath{H}\xspace}
\newcommand{\pwd}{\mbox{\textit{pwd}}\xspace}
\newcommand{\pw}{\textsc{Catena}\xspace}
%\newcommand{\pwl}{\ensuremath{\textsc{Catena}_{\FL,\hf}}\xspace}
\newcommand{\pwl}{\ensuremath{\textsc{Catena}}\xspace}
%\newcommand{\pwlk}{\ensuremath{\textsc{Catena}^K_{\FL,\hf}}\xspace}
\newcommand{\pwlk}{\ensuremath{\textsc{Catena}^K}\xspace}
\newcommand{\pwbrg}{\textsc{\pw-BRG}\xspace}
\newcommand{\pwdbg}{\textsc{\pw-DBG}\xspace}
\newcommand{\pwhead}{{\large \textsc{Catena}}\xspace}
\newcommand{\pwtitle}{{\LARGE \textsc{Catena}}\xspace}
\newcommand{\pwkg}{\textsc{Catena-KG}\xspace}
\newcommand{\pwax}{\textsc{Catena-Axungia}\xspace}
\newcommand{\pwvar}{\textsc{Catena-Variants}\xspace}
\newcommand{\pwkghead}{{\large \textsc{Catena-KG}}\xspace}

\newcommand{\lmaa}{\pw}
\newcommand{\lmaashort}{\ensuremath{\pw\text{-}\lambda}\xspace}
\newcommand{\lmaatic}{\ensuremath{\pw'\text{-}\lambda}\xspace}
\newcommand{\lamfunc}{\ensuremath{\mbrhfa(\cdot,\cdot)}\xspace}
\newcommand{\FL}{\ensuremath{F}\xspace}
\newcommand{\RL}{\ensuremath{\Gamma}\xspace}
\newcommand{\MHF}{\ensuremath{flap}\xspace}
\newcommand{\seedrand}{\ensuremath{seed\_rand}\xspace}
\newcommand{\updaterand}{\ensuremath{update\_rand}\xspace}


% additional stuff to include pseudocode

%%%%%%%%%%%%%%%%%%%%%%%%%%%%%%%%%%%%%%%%%%%%%%%%%%
% Catena
%%%%%%%%%%%%%%%%%%%%%%%%%%%%%%%%%%%%%%%%%%%%%%%%%%

\newcommand{\catena}{\ensuremath{\mbox{\textsc{Catena-}}\lambda}\xspace}
%\newcommand{\catenabrg}{\textsc{Catena-BRG}\xspace}
%\newcommand{\catenadbg}{\textsc{Catena-DBG}\xspace}
\newcommand{\catenaant}{\textsc{Catena-Dragonfly}\xspace}
\newcommand{\catenaantfull}{\textsc{Catena-Dragonfly-Full}\xspace}
\newcommand{\catenabut}{\textsc{Catena-Butterfly}\xspace}
\newcommand{\catenabutfull}{\textsc{Catena-Butterfly-Full}\xspace}
\newcommand{\catenawl}{\textsc{Catena}\xspace}
\newcommand{\catenakg}{\ensuremath{\mbox{\texttt{Catena-KG}}}\xspace}
\newcommand{\lbrg}{\ensuremath{\lambda\mbox{\texttt{-BRG}}}\xspace}
\newcommand{\lbrh}{\ensuremath{\lambda\mbox{\texttt{-BRH}}}\xspace}

\newcommand{\romix}{\texttt{ROMix}\xspace}
\newcommand{\romixmc}{\texttt{ROMixMC}\xspace}
\newcommand{\scrypt}{\texttt{scrypt}\xspace}

\newcommand{\header}{\ensuremath{H}\xspace}
\newcommand{\Garlic}{\ensuremath{G}\xspace}
\newcommand{\garlic}{\ensuremath{g}\xspace}
\newcommand{\inp}{\ensuremath{x}\xspace}
\newcommand{\salt}{\ensuremath{s}\xspace}
\newcommand{\tweak}{\ensuremath{U}\xspace}
\newcommand{\node}[1]{\ensuremath{v_{#1}}\xspace}
\newcommand{\nodes}{\ensuremath{\mathfrak{V}}\xspace}
\newcommand{\graph}{\ensuremath{\Pi(\nodes,\edges)}\xspace}
\newcommand{\ggraph}{\ensuremath{\Pi_\garlic^\lambda(\nodes,\edges)}\xspace}

\renewcommand{\H}{\ensuremath{\mathcal{H}}\xspace}
\renewcommand{\algorithmicrequire}{\textbf{Input:}}
\renewcommand{\algorithmicensure}{\textbf{Output:}}

\newcommand{\mbrgfax}{\ensuremath{\lambda\text{-BRG}}\xspace}
\newcommand{\mbrhfax}{\ensuremath{\lambda\text{-BRH}}\xspace}
\newcommand{\mbrhfaxm}{\ensuremath{\lambda\text{-}BRH}\xspace}
\newcommand{\mbrhfa}{\ensuremath{BRH_\lambda^g}\xspace}


\newcommand{\testentry}[3]{\textbf{#1} & \texttt{#2} & (#3 octets)\\}

\newcommand{\blake}{BLAKE2b\xspace}
\newcommand{\blakefast}{BLAKE2b-1\xspace}

\newcommand{\length}[1]{\ensuremath{|#1|}\xspace}

\newcommand{\glow}{\ensuremath{\garlic_{\text{low}}}\xspace}
\newcommand{\ghigh}{\ensuremath{\garlic_{\text{high}}}\xspace}


\newcommand{\catenabrg}{\textsc{Catena-BRG}\xspace}
\newcommand{\catenadbg}{\textsc{Catena-DBG}\xspace}
\newcommand{\hfinit}{\ensuremath{H_{\text{init}}}\xspace}
\newcommand{\hffirst}{\ensuremath{H_{\text{first}}}\xspace}

\newcommand{\overviewPDF}{\includegraphics[height=0.85\textheight]{Overview.eps}}


%%%%%%%%%%%%%%%%%%%%%%%%%%%%%%%%%%%%%%%%%%%%%%%%%%
% pseudo code

\newcommand{\pscatena}{\
    \begin{algorithm}[H] \caption*{\pw}
      \begin{algorithmic}[1]
        \REQUIRE $pwd$, $t$, $s$, \glow, \ghigh, $m$, $\gamma$
        \ENSURE $x$
        \STATE $x \gets \hf(t \sep pwd \sep s)$
        \STATE $x \gets \MHF(\lceil \glow/2\rceil,x,\gamma)$
        \label{line:mhf}
        \FOR{$g=\glow,\ldots,\ghigh$}
          \STATE $x \gets \MHF(g,x \sep 0^*,\gamma)$
          \label{line:tabula-catena}
          \STATE $x \gets \hf(g \sep x)$
          \STATE $x \gets truncate(x,m)$
        \ENDFOR
        \RETURN $x$
      \end{algorithmic}
      \label{alg:catena}
    \end{algorithm}}

\newcommand{\psflap}{\
    \begin{algorithm}[H]
    \caption*{\MHF}
      \begin{algorithmic}[1]
        \REQUIRE $g$, $x$, $\gamma$
        \ENSURE $x$
        \STATE $(v_{-2},v_{-1}) \gets \hfinit(x)$
        \label{line:init1}
        \FOR{$i=0 ,\ldots, 2^g-1$}
        \label{line:loop1-1}
          \STATE $v_i \gets H'(v_{i-1}\sep v_{i-2})$
          \label{line:loop1-2}
        \ENDFOR
        \label{line:init2}
        \STATE $v \gets \RL(g,v,\gamma)$
        \label{line:public_input}
        \STATE $x \gets \FL(v)$
        \label{line:memory_hard}
        \STATE \textbf{return} $x$
      \end{algorithmic}
      \label{alg:mhf}
    \end{algorithm}}

\newcommand{\pshinit}{\
    \begin{algorithm}[H]
    \caption*{\hfinit}
      \begin{algorithmic}[1]
        \REQUIRE $x$
        \ENSURE $v_{-2},v_{-1}$
        \STATE $\ell = 2\cdot k/n$
        \FOR{$i=0,\ldots,\ell-1$}
        \label{line:init-large-1}
          \STATE $w_i \gets \hf(i\sep x)$
          \label{line:hfinit-hf}
        \ENDFOR
        \STATE $v_{-2} \gets (w_0\sep\ldots\sep\ w_{\ell/2-1})$
        \STATE $v_{-1} \gets (w_{\ell/2}\sep\ldots\sep\ w_{\ell-1})$
        \STATE \textbf{return} $(v_{-2},v_{-1})$
        \label{line:init-large-2}
      \end{algorithmic}
      \label{alg:hfinit}
    \end{algorithm}}

\newcommand{\psciupdate}{\
    \begin{algorithm}[H]
    \caption*{Client-Independent Update}
      \begin{algorithmic}[1]
        \REQUIRE $h, g_{high}, \lambda, m, \gamma, {g'}_{high}$
        \ENSURE $h'$
        \STATE $h' \gets h$
        \FOR{$g=g_{high},\ldots,{g'}_{high}$}
          \STATE $h' \gets flap(g,h' \sep 0^*,\gamma,\lambda)$
          \STATE $h' \gets H(g\sep h')$
          \STATE $h' \gets truncate(h',m)$
        \ENDFOR
        \STATE \textbf{return} $h'$
      \end{algorithmic}
    \end{algorithm}}

\newcommand{\pskg}{\
    \begin{algorithm}[H]
    \caption*{\pw-KG}
      \begin{algorithmic}[1]
        \REQUIRE $pwd, t', s, g_{low}, g_{high}, \gamma, \ell, \mathcal{I},
        \lambda$
        \ENSURE $k$
        \STATE $x \gets \pw(pwd, t', s, g_{low}, g_{high}, n, \gamma)$
        \STATE $k \gets \emptyset$
        \FOR{$i=1,\ldots,\lceil\ell/m\rceil$}
          \STATE $k \gets k \sep H(i \sep \mathcal{I} \sep \ell \sep x)$
        \ENDFOR
        \STATE \textbf{return} $(k, \ell)$
      \end{algorithmic}
    \end{algorithm}}

\usepackage{color}
\definecolor{gray}{rgb}{0.4,0.4,0.4}
\definecolor{darkblue}{rgb}{0.0,0.0,0.6}
\definecolor{cyan}{rgb}{0.0,0.6,0.6}

\lstset{
  basicstyle=\ttfamily,
  columns=fullflexible,
  showstringspaces=false,
  commentstyle=\color{gray}\upshape
}

\lstdefinelanguage{XML}
{
  morestring=[b]",
  morestring=[s]{>}{<},
  morecomment=[s]{<?}{?>},
  stringstyle=\color{black},
  identifierstyle=\color{darkblue},
  keywordstyle=\color{cyan},
  morekeywords={xmlns,version,type}% list your attributes here
}

\usetheme{CambridgeUS}


\definecolor{midnightblue}{rgb}{0,.25,.5}
\definecolor{darkblue}{rgb} {0.125,0.25, 0.625}
\definecolor{darkgreen}{rgb}{0.125,0.625,0.25}
%\definecolor{darkred}{rgb}  {0.625,0.125,0.25}
\definecolor{darkred}{rgb}  {0.0,0.4,0.59}

\mode<presentation>
{
  \useinnertheme{rectangles}
  \setbeamertemplate{navigation symbols}{}
}

\setbeamercolor{footlinecolorl}{fg=black,bg=lightgray}
\setbeamercolor{footlinecolor}{fg=black,bg=gray}
\setbeamercolor{footlinecolord}{fg=black,bg=darkgray}
\setbeamercolor{block title}{bg=darkred,fg=white}

\setbeamertemplate{itemize item}{\color{darkred}$\blacksquare$}
\setbeamertemplate{itemize subitem}{\color{darkred}$\blacktriangleright$}
\setbeamercolor{item projected}{bg=darkred}
\setbeamertemplate{enumerate items}[default]
\setbeamercolor{enumerate item}{fg=darkred}
\setbeamercolor{enumerate subitem}{fg=darkred}
\setbeamercolor{enumerate subsubitem}{fg=darkred}

\setbeamertemplate{footline}{%
\hbox{%
\begin{beamercolorbox}[wd=.40\paperwidth,ht=4.25ex,left,leftskip=3ex]{author in head/foot}%
    \vbox to4.25ex{\vfil\hbox{\usebeamerfont{author in head/foot} \insertshortauthor}\vfil}%
\end{beamercolorbox}%
\begin{beamercolorbox}[wd=.25\paperwidth,ht=4.25ex,center]{title in head/foot}%
    \vbox to4.25ex{\vfil\hbox{\usebeamerfont{date in
    head/foot}\insertshorttitle{}}\vfil}%
\end{beamercolorbox}%
\begin{beamercolorbox}[wd=.05\paperwidth,ht=4.25ex,center]{title in head/foot}%
    \vbox to4.25ex{\vfil\hbox{\usebeamerfont{title in head/foot}\insertframenumber/\inserttotalframenumber}\vfil}%
\end{beamercolorbox}%
\begin{beamercolorbox}[wd=.30\paperwidth,ht=4.25ex,right,rightskip=3ex]{date in head/foot}%
    \vbox to4.25ex{\vfil\hbox{\insertshortdate{}}\vfil}%
\end{beamercolorbox}}%
}

%\setbeamertemplate{footline}{
%  \quad \tiny \insertshortauthor \hfill \insertshorttitle \qquad \hfill \insertshortdate\ \qquad\qquad
%}


% ---------------------------------------------------------------------
% Title
% ---------------------------------------------------------------------

\title[Green Development]{GreenDev}
%\subtitle{Subtitle (project name)}
\author[M. Weber]{Max Weber}
\institute[Bauhaus-Universität Weimar]{}
\date[\today]{\today}

\raggedright
\AtBeginSection{\frame{\sectionpage}}

% ---------------------------------------------------------------------

\begin{document}

% ---------------------------------------------------------------------
% Content
% ---------------------------------------------------------------------

\maketitle

%\begin{frame}{Outline}
%\tableofcontents
%\end{frame}

% ---------------------------------------------------------------------
\section{Catena}
% ---------------------------------------------------------------------

\begin{frame}{Catena General}
  \begin{itemize}
    \item password hashing framework
    \item also possible: proof of work, server relief, key derivation, keyed password hashing, client independent update
  \end{itemize}
\end{frame}

\begin{frame}{Catena Structure}
  \begin{minipage}[H]{0.55\linewidth}
    \begin{itemize}
      \item BLAKE2b is a fast and secure cryptographic hash function
      \item default of H is BLAKE2b
      \item default of H' is BLAKE2b-1
    \end{itemize}
  \end{minipage}
  \hfill
  \begin{minipage}[H]{0.4\linewidth}
    \includegraphics[height=0.85\textheight]{./images/overview.pdf}
  \end{minipage}
\end{frame}

\begin{frame}{Catena Parameter}
  \begin{minipage}[H]{0.55\linewidth}
    parameter to start Catena
    \begin{itemize}
      \item pwd: password
      \item t: tweak, additional salt
      \item s: salt
      \item $g_{low}=g_{high}$: garlic, cost parameter (time)
      \item m: output length
      \item $\gamma$: gamma, cost parameter (memory)
    \end{itemize}
  \end{minipage}
  \hfill
  \begin{minipage}[H]{0.4\linewidth}
    \pscatena
  \end{minipage}
\end{frame}

\begin{frame}{Catena Parameter}
  \begin{minipage}[H]{0.55\linewidth}
    tweak:
    \begin{itemize}
      \item allows to differentiate between different versions of Catena
      \item $V$: unique version identifier
      \item $d$: mode of \pw (i.e. $d=0$ for password scrambler)
      \item $\lambda$: defines with $g$ dimensions of graph
      \item $m$: output length
      \item $|s|$: salt length
      \item $AD$: optional associated data
    \end{itemize}
  \end{minipage}
  \hfill
  \begin{minipage}[H]{0.4\linewidth}
    \[
      t \gets H(V) \sep d\sep\lambda \sep m \sep |s| \sep H(\mbox{AD})
    \]
  \end{minipage}
\end{frame}

\begin{frame}{Configuration}
  user inputs:
  \begin{itemize}
    \item pwd: password
    \item s: salt
    \item m: output length
  \end{itemize}
\end{frame}

\begin{frame}{Configuration}
  Catena parameter:
  \begin{itemize}
    \item (256!) $g_{low}=g_{high}$: garlic (defined by default)
    \item (256!) $\lambda$: lambda (defined by default)
    
    \item (byte[]) $V$: unique version identifier (defined by default)
    \item (256) $d$: mode of Catena (defined by default)
    \item (byte[]) $AD$: optional associated data

    \item (byte[]) $\gamma$: gamma (only required if GAMMA)
    \item (2) HASH: (defined by default)
    \item (2) GAMMA: (defined by default)
    \item (4) GRAPH: (defined by default)
    \item (2) PHI: (defined by default)
  \end{itemize}
\end{frame}

\begin{frame}{Configuration}
  Catena parameter:
  \begin{itemize}
    \item $(256!)^3 * 2^5 * byte[] ^ 3 = 536,870,912 * byte[] ^ 3$
  \end{itemize}
  h2 parameter:
  \begin{itemize}
    \item $2^{15} = 32,768$
  \end{itemize}
\end{frame}

\begin{frame}{$BRG^g_\lambda$}
  \begin{minipage}[H]{0.48\linewidth}
    \begin{figure}
      \includegraphics[width=\textwidth]{images/brg_3-eps-converted-to.pdf}\\
        \caption{$BRG^3_1$}
    \end{figure}
  \end{minipage}
  \hfill%
  \begin{minipage}[H]{0.48\linewidth}
    \begin{itemize}
      \item g denotes the logarithm of the memory required per level
      \item $\lambda$ the number of levels, i.e.\ the depth
    \end{itemize}
  \end{minipage}
\end{frame}

\begin{frame}{$DBG^g_\lambda$}
  \begin{minipage}[H]{0.48\linewidth}
    \begin{figure}
      \includegraphics[width=\textwidth]{images/superconcentrator-nl-eps-converted-to.pdf}
      \caption{$DBG^3_1$}
    \end{figure}
  \end{minipage}
  \hfill%
  \begin{minipage}[H]{0.48\linewidth}
    \begin{itemize}
      \item g denotes the logarithm of the memory required per level
      \item $\lambda$ the number of levels, i.e.\ the depth
    \end{itemize}
  \end{minipage}
\end{frame}

% ---------------------------------------------------------------------
\section{Profile Script}
% ---------------------------------------------------------------------

\begin{frame}{Profile Script}
  \begin{itemize}
    \item python script for executing tests while profiling
    \item generates project parameter
    \item generates program and vm parameter
    \item saves all results structured
    \item defines maximum amount of configurations
    \item need to be reworked ...
  \end{itemize}
\end{frame}

\begin{frame}{Other Profiler}
  \begin{itemize}
    \item To Try:
    \begin{itemize}
      \item Java Mission Control \url{http://www.adam-bien.com/roller/abien/entry/java_mission_control_development_pricing}
      \item honest profiler \url{https://github.com/jvm-profiling-tools/honest-profiler/wiki/How-to-build}
      \item patty \url{http://patty.sourceforge.net/}
      \item statsd-jvm-profiler \url{https://github.com/etsy/statsd-jvm-profiler}
      \item Simple java profiler \url{https://github.com/mitallast/simple-java-profiler}
    \end{itemize}
    \item Others:
    \begin{itemize}
      \item YourKit (License necessary to extract data (15 Days trial))
      \item JProfiler (License necessary to run it)
      \item VisualVM (only GUI)
    \end{itemize}
  \end{itemize}
\end{frame}

% ---------------------------------------------------------------------
\section{Performance Visualisation}
% ---------------------------------------------------------------------

\begin{frame}{Different Modes}
  \begin{itemize}
    \item max of one cfg
    \item use absolute time values for comparison of all configuration
    \item mean of all cfg's
    \item max of all cfg's
    \item input sensitivity (abs(mean,median))
    \item standard deviation (measure of how far the sample spreads on average by the arithmetic mean)
  \end{itemize}
\end{frame}

% ---------------------------------------------------------------------
\section{Related Work}
% ---------------------------------------------------------------------

\begin{frame}{Spectrum-based Energy Leak Localization}
  \begin{itemize}
    \item from October 2014
    \item "Universidade do Minho"
    \item relation of energy consumption to source code
    \item identify and mark "abnormal energy consumption"
    \item spectrum is a "set of run-time execution data of a program"
    \item they used SFL (Spectrum-based Fault Localization) to identify those parts which are more likely to being faulty
  \end{itemize}
\end{frame}

\begin{frame}{Spectrum-based Energy Leak Localization}
  Similarities to my thesis:
  \begin{itemize}
    \item analyses on function level
    \item heavy processes
    \item many tests 
    \item different configurations and different inputs
    \item also coloring of source code (only red)
  \end{itemize}
  Differences:
  \begin{itemize}
    \item AST manipulation to append logging functionality
    \item pause the program execution to force vm to lower the CPU impact on next functions
  \end{itemize}
\end{frame}

\begin{frame}{Related Paper}
  \begin{itemize}
    \item On the Accuracy of Software-Based Energy Estimation Techniques

    \item Proactive Energy-Aware System Software Design with SEEP

    \item Unit Testing of Energy Consumption of Software Libraries

    \item Powermanagement architecture of the intel microarchitecture code-named sandy bridge
  \end{itemize}
\end{frame}

% ---------------------------------------------------------------------
\section{Next Steps}
% ---------------------------------------------------------------------

\begin{frame}{Later}
  \begin{itemize}
    \item abstract
    \item feature model of projects
  \end{itemize}
\end{frame}

%-----------------------------------------------------------------------
% Appendix
% ---------------------------------------------------------------------

\appendix
\end{document}
